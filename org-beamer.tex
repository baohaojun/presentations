% Created 2018-11-26 月 09:51
% Intended LaTeX compiler: pdflatex
\documentclass[presentation]{beamer}
\usepackage[utf8]{inputenc}
\usepackage[T1]{fontenc}
\usepackage{graphicx}
\usepackage[export]{adjustbox}
\usepackage{lmodern}
\usepackage{grffile}
\usepackage{longtable}
\usepackage{wrapfig}
\usepackage{rotating}
\usepackage[normalem]{ulem}
\usepackage{amsmath}
\usepackage{textcomp}
\usepackage{amssymb}
\usepackage{capt-of}
\usepackage{hyperref}
\usepackage{minted}
\usetheme{default}
\author{Eric S Fraga}
\date{2010-10-02 Sat}
\title{Writing Beamer presentations in org-mode}
\hypersetup{
 pdfauthor={Eric S Fraga},
 pdftitle={Writing Beamer presentations in org-mode},
 pdfkeywords={},
 pdfsubject={},
 pdfcreator={Emacs 26.1 (Org mode 9.1.9)}, 
 pdflang={English}}
\begin{document}

\maketitle
\begin{frame}{Outline}
\tableofcontents
\end{frame}


\section{Introduction}
\label{sec:orgcba6fdb}

Beamer is a \LaTeX{} package for writing presentations.  This documents
presents a simple introduction to preparing beamer presentations using
org-mode in Emacs.

This documents assumes that the reader is already acquainted with org-mode
itself and with \alert{exporting} org-mode documents to \LaTeX{}.  There are tutorials
and references available for both org-mode itself, for
\LaTeX{} exporting, and
for
Beamer
Class Export.  The document also assumes that the reader understands the
notation for
Emacs
keybindings.

\section{First steps}
\label{sec:orgce18ac7}
\subsection{The export template}
\label{sec:orgd16929e}
Starting with an empty file called \texttt{presentation.org} [1], say, the
first step is to insert the default org export template (\texttt{C-c C-e t}
with the default keybindings). This will generate something that looks
like this (some specific entries will vary):

\begin{verbatim}
#+TITLE:     Writing Beamer presentations in org-mode
#+AUTHOR:    Eric S Fraga
#+EMAIL:     e.fraga@ucl.ac.uk
#+DATE:      2010-03-30 Tue
#+DESCRIPTION:
#+KEYWORDS:
#+LANGUAGE:  en
#+OPTIONS:   H:3 num:t toc:t \n:nil @:t ::t |:t ^:t -:t f:t *:t <:t
#+OPTIONS:   TeX:t LaTeX:t skip:nil d:nil todo:t pri:nil tags:not-in-toc
#+INFOJS_OPT: view:nil toc:nil ltoc:t mouse:underline buttons:0 path:http://orgmode.org/org-info.js
#+EXPORT_SELECT_TAGS: export
#+EXPORT_EXCLUDE_TAGS: noexport
#+LINK_UP:
#+LINK_HOME:
\end{verbatim}

In this default template, you will want to modify, at the very least,
the title, as I have done, as this will be used as the title of your
presentation.  It will often be useful to modify some of the \LaTeX{}
export options, most commonly the \texttt{toc} option for generating a table
of contents.  For this document, and the associated sample
presentation, I have left all options as they are by default according
to the template.
\subsection{Beamer specific settings}
\label{sec:org279ab00}
As well as the general options provided by the template, there are
Beamer specific options.  The following options are what I use:

\begin{verbatim}
#+startup: beamer
#+LaTeX_CLASS: beamer
#+LaTeX_CLASS_OPTIONS: [bigger]
\end{verbatim}

The first line enables the Beamer specific commands for org-mode (more
on this below); the next two tell the \LaTeX{} exporter to use the
Beamer class and to use the larger font settings[2].

\subsection{Outline levels for frames (slides)}
\label{sec:orga956ce6}

The following line specifies how org headlines translate to the Beamer
document structure.

\begin{verbatim}
#+BEAMER_FRAME_LEVEL: 2
\end{verbatim}

A Beamer presentation consists of a series of slides, called \emph{frames}
in Beamer.  If the option shown above has a value of 1, each top level
headline will be translated into a frame.  Beamer, however, also makes
use of \LaTeX{} sectioning to group frames.  If this appeals, setting
the option to a value of 2 tells org to export second level headlines
as frames with top level headlines translating to sections.
\subsection{Column view for slide and block customisation}
\label{sec:org254f471}
The final line that is useful to specify to set up the presentation is

\begin{verbatim}
#+COLUMNS: %40ITEM %10BEAMER_env(Env) %9BEAMER_envargs(Env Args) %4BEAMER_col(Col) %10BEAMER_extra(Extra)
\end{verbatim}

The purposes of this line is to specify the format for the special
interface that org-mode provides to control the layout of individual
slides.  More on this below.

Once all of the above has been set up, you are ready to write your
presentation.

\section{The slides}
\label{sec:org35600e1}

Each slide, or \emph{frame} in Beamer terminology, consists of a title and
the content.  The title will be derived from the outline headline text
and the content will simply be the content that follows that
headline.  A few example slides are presented below.  These will only
cover the basic needs; for more complex examples and possible
customisations, I refer you to the detailed manual.

\subsection{A simple slide}
\label{sec:org2944955}
The simplest slide will consist of a title and some text.  For instance,

\begin{verbatim}
* Introduction
** A simple slide
This slide consists of some text with a number of bullet points:

- the first, very @important@, point!
- the previous point shows the use of the special markup which
  translates to the Beamer specific /alert/ command for highlighting
  text.


The above list could be numbered or any other type of list and may
include sub-lists.
\end{verbatim}

defines a new section, \emph{Introduction}, which has a slide with title
\emph{A simple slide} and a three item list.  The result of this with the
settings defined above, mostly default settings, will generate a slide
that looks like this:

\begin{figure}[htbp]
\centering
\includegraphics[width=.9\linewidth]{../../images/org-beamer/a-simple-slide.png}
\caption{Simple slide exported from org to \LaTeX{} using beamer}
\end{figure}

\subsection{A more complex slide using blocks}
\label{sec:org8a3463a}

Beamer has the concept of block, a set of text that is logically
together but apart from the rest of the text that may be in a slide.
How blocks are presented will depend on the Beamer theme used
(customisation in general and choosing the theme specifically are
described below).

There are many types of blocks.  The following

\begin{verbatim}
** A more complex slide
This slide illustrates the use of Beamer blocks.  The following text,
with its own headline, is displayed in a block:
*** Org mode increases productivity                               :B_theorem:
    :PROPERTIES:
    :BEAMER_env: theorem
    :END:
    - org mode means not having to remember LaTeX commands.
    - it is based on ascii text which is inherently portable.
    - Emacs!

    \hfill \(\qed\)
\end{verbatim}

creates a slide that has a title (the headline text), a couple of
sentences in paragraph format and then a \emph{theorem} block (in which I
prove that org increases productivity).  The theorem proof is a list
of points followed a bit of \LaTeX{} code at the end to draw a fancy
\emph{end of proof} symbol right adjusted.

You will see that there is an org properties \emph{drawer} that tells org
that the text under this headline is a block and it also specifies the
type of block.  You do not have to enter this text directly yourself;
org-mode has a special beamer sub-mode which provides an easy to use
method for specifying block types (and columns as well, as we shall
see in the next section).

To specify the type of block, you can type \texttt{C-c C-b} [3].  This brings up
a keyboard driven menu in which you type a single letter to select the
option you wish to apply to this headline.  For the above example, I
typed \texttt{C-c C-b t}.  The options selected in this manner are also shown
as \emph{tags} on the headline.  However, note that the tag is for display
only and has no direct effect on the presentation.  You cannot change
the behaviour by changing the tag; it is the property that controls
the behaviour.

\subsection{Slides with columns}
\label{sec:org41b12e4}

The previous section introduced the special access keys (\texttt{C-c C-b})
for defining blocks.  This same interface allows you to define
columns.  A headline, as the text that follows it, can be in a block,
in a column, or both simutaneously.  The \texttt{|} option will define a
column.  The following

\begin{verbatim}
** Two columns

*** A block                                           :B_ignoreheading:BMCOL:
    :PROPERTIES:
    :BEAMER_env: ignoreheading
    :BEAMER_col: 0.4
    :END:
    - this slide consists of two columns
    - the first (left) column has no heading and consists of text
    - the second (right) column has an image and is enclosed in an
      @example@ block

*** A screenshot                                            :BMCOL:B_example:
    :PROPERTIES:
    :BEAMER_col: 0.6
    :BEAMER_env: example
    :END:
,    #+ATTR_LATEX: width=\textwidth
    [[//../../images/org-beamer/a-simple-slide.png]]
\end{verbatim}

defines a two column slide.  As the text in the slide says, the left
column is a list and the right one is an image.  The left column's
headline text is ignored, specified by \texttt{C-c C-b i} which tells org to
\alert{ignore} the headline text completely.  The column on the right
however is placed with an \emph{example} block (whose appearance will
depend on the Beamer theme).

The columns also have widths.  By default, these widths are the
proportion of the page width to use so I have specified 40\% for the
left column and 60\% for the right one.

The image in the right column is inserted simply by specifying a link
to the image file with no descriptive text.  I have added an attribute
to the image (see the \texttt{\#+ATTR\_LATEX} line above) to tell \LaTeX{} to scale
the image to the full width of the column (\texttt{\textbackslash{}textwidth}).

\subsection{Using Babel}
\label{sec:org2fe1fea}
One of my main uses for Beamer is the preparation of slides for
teaching.  I happen to teach Octave to engineering students.  Org
provides the Babel framework for embedding code within org
files.  For teaching, this is an excellent tool for presenting codes
and the results of evaluating those codes.

For instance, the following code:
\begin{verbatim}
** Babel
   :PROPERTIES:
   :BEAMER_envargs: [t]
   :END:
*** Octave code                                                      :BMCOL:B_block:
    :PROPERTIES:
    :BEAMER_col: 0.45
    :BEAMER_env: block
    :END:
#+name: octaveexample
#+begin_src octave :results output :exports both
A = [1 2 ; 3 4]
b = [1; 1];
x = A\b
#+end_src

*** The output                                               :BMCOL:B_block:
    :PROPERTIES:
    :BEAMER_col: 0.4
    :BEAMER_env: block
    :BEAMER_envargs: <2->
    :END:

#+results: octaveexample
#+begin_example
A =

   1   2
   3   4

x =

  -1
   1

#+end_example

\end{verbatim}

will generate a slide with two blocks and a pause between the display
of each of the two blocks:

\begin{figure}[htbp]
\centering
\includegraphics[width=.9\linewidth]{../..//images/org-beamer/babel-octave.png}
\caption{The use of babel for code display and execution}
\end{figure}

\section{Customisation}
\label{sec:org68e4578}

Org has a very large number of customisable aspects.  Although
daunting at first, most options have defaults that are suitable for
most people using org initially.  The same applies to the Beamer
export support.  However, there are some options which many will soon
wish to change.

\subsection{Beamer theme}
\label{sec:orgb7df606}

Beamer has a large number of themes and I simply refer the reader to
the manual or the Web to find what themes are available and what they
look like.  When you have chosen a theme, you can tell org to use it
by inserting some direct \LaTeX{} code into the \emph{preamble} of the
document, the material that comes before the first headline.  For
instance, adding this line
\begin{verbatim}
#+latex_header: \mode<beamer>{\usetheme{default}}
\end{verbatim}
to the preamble after the beamer font size
option described above will produce a presentation that looks very
different from the default (with no other changes required!):

\begin{figure}[htbp]
\centering
\includegraphics[width=.9\linewidth]{../../images/org-beamer/two-column-slide-madrid-style.png}
\caption{Two column slide with the Madrid Beamer theme}
\end{figure}

\subsection{Table of contents}
\label{sec:org7f1748b}

The default \texttt{toc:t} option generated by the export template command
(\texttt{C-c C-e t}) indicates that a table of contents will be generated.
This will create a slide immediately after the title slide which will
have the list of sections in the beamer document.  Please note that if
you want this type of functionality, you will have to specify the
\texttt{BEAMER-FRAME-LEVEL} to be 2 instead of 1 as indicated above.

Furthermore, if you have decided to use sections, it is possible to
have Beamer automatically place a table of contents slide before the
start of each section with the new section highlighted.  This is
achieved by inserting the following \LaTeX{} code, again in the
preamble:
\begin{verbatim}
#+latex_header: \AtBeginSection[]{\begin{frame}<beamer>\frametitle{Topic}\tableofcontents[currentsection]\end{frame}}
\end{verbatim}

\subsection{Column view for slide and block customisation}
\label{sec:org3feb8e4}

In an early section of this document, I described a magical
incantation!  This incantation defines the format for viewing org
property information in column mode.  This mode allows you to easily
adjust the values of the properties for any headline in your
document.  This image shows the type of information you can see at a
glance in this mode:

\begin{figure}[htbp]
\centering
\includegraphics[width=.9\linewidth]{../../images/org-beamer/column-view.png}
\caption{Column view of presentation showing special block environments and column widths.}
\end{figure}

We can see the various blocks that have been defined as well as any
columns (implicit by the presence of a column width).  By moving to
any of these column entries displayed, values can be added, deleted or
changed easily.  Please read the full org Beamer manual for details.
\end{document}
